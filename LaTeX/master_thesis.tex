\documentclass[12pt, a4paper]{article}
\usepackage[left=3cm, right = 2cm, vmargin=2cm]{geometry}
\usepackage{microtype}
\usepackage{graphicx}
\usepackage{hyperref}
\usepackage[english]{babel}
\usepackage{setspace}
\usepackage{xcolor}
\usepackage{multicol}
\usepackage{float}
\usepackage{amsmath}
\usepackage{amssymb}
\usepackage{mathtools}
\usepackage{titlesec}
\usepackage{bbm}
\usepackage[font=small,skip=2pt]{caption}
\usepackage[
backend=biber,
style=authoryear,
]{biblatex}
\usepackage{amsthm}% http://ctan.org/pkg/amsthm

\DeclarePairedDelimiter\ceil{\lceil}{\rceil}
\DeclarePairedDelimiter\floor{\lfloor}{\rfloor}

\newtheoremstyle{MAstyle}
{\topsep} % Space above
{\topsep} % Space below
{} % Body font
{} % Indent amount
{\bfseries} % Theorem head font
{\newline} % Punctuation after theorem head
{.5em} % Space after theorem head
{} % Theorem head spec (can be left empty, meaning `normal')

\theoremstyle{MAstyle} \newtheorem{assumption}{Assumption}[section]
\theoremstyle{MAstyle} \newtheorem{definition}{Definition}[section]

\titleformat*{\section}{\Large\bfseries}
\titleformat*{\subsection}{\large\bfseries}

\addbibresource{thesis_bibliography.bib}
\onehalfspacing
\parindent=0pt

\begin{document}
	
	\title{{\huge Bugni and Horowitz (2021) Permutation Tests\\ for the Equality of Distributions of\\ Functional Data}}
	\date{}
	\maketitle
	\thispagestyle{empty}
	\vspace{1.5 cm}
	\begin{center}
		
		\Large
		Master's Thesis presented to the\\
		Department of Economics at the\\
		Rheinische Friedrich-Wilhelms-Universität Bonn
		\vspace{1.5cm}

		\large
		In Partial Fulfillment of the Requirements for the Degree of\\
		Master of Science (M.Sc.)
		
		\vspace{3cm}
		
		Supervisor: Prof. Dr. Dominik Liebl
		
		\vspace{3cm}
		
		Submitted in June 2022 by: \\
		Jakob R. Juergens\\
		Matriculation Number: 2996491
	\end{center}
	
	\newpage
	
	\thispagestyle{empty}
	\tableofcontents
	
	\newpage
	\pagenumbering{arabic}
	
	\section{Introduction}
		In modern economics, it is becoming more and more common to use data measured at a very high frequency. As the frequency of observing a variable increases, it often becomes more natural to view the data not as a sequence of distinct observation points but as a smooth curve that describes the variable over time.
		This idea, to think of observations as measurements of a continuous process, is the motivating thought behind functional data analysis. Functional data analysis is a branch of statistics that has its beginnings in the 1940s and 1950s in the works of Ulf Grenander and Kari Karhunen. It gained traction during the following decades and focused more on possible applications during the 1990s. In economics, functional data analysis is still a relatively exotic field, but it is beginning to become more established, which can be seen in the works of, for example, {\color{red} hier Autoren einfuegen}.\\
		
		A typical question in economics is whether observations from two or more data sets, e.g., data generated by treatment and control groups, are systematically different across groups. In statistical terms, this can be formulated as whether observations in both data sets can be seen as if the same stochastic process generated them.		
		This question can also occur in functional data analysis, where each observation in a data set is itself a smooth curve. \cite{bugni_permutation_2021} develop a permutation test that tries to answer this question by combining two distinct test statistics. To explore their approach, it is first necessary to introduce some theoretical concepts.
		Section \ref{FDA} introduces the necessary concepts from functional data analysis. Section \ref{CvM_Tests} explores the theory around Cram\'{e}r-von Mises tests. Section \ref{Multiple_Testing} introduces the Bonferroni Correction for multiple testing problems and Section \ref{Permutation_Tests} finally introduces the necessary background in Permutation Testing.
		After explaining these concepts, section \ref{Bugni_Horowitz_2021} focuses on the test developed in \cite{bugni_permutation_2021} for the case of a two sample test. Section \ref{Simulation_Study} replicates the results from the simulation study in the paper and section \ref{Application} explores their usefulness in an application to {\color{red} Thema der Anwendung}.
	
	\section{Functional Data Analysis}\label{FDA}
		The overarching concept of functional data analysis is to incorporate observations that are functional in nature. In this context, a functional observation can often be understood as a smooth curve. A classical example of this is shown in Figure \ref{growth_curves}. It presents data provided in the R package \textit{fda}\footcite{fda} and shows growth curves of 93 humans up to the age of 18.
		\begin{figure}[H]
			\includegraphics[width = \textwidth]{../Graphics/growth\_curves.PDF}
			\caption{Human Growth Curves up to the Age of 18}
			\label{growth_curves}
		\end{figure}
		Even though the measurements were taken at discrete ages, it is clear that each human has a height at every point in time. The data points are only measurements of this continuous curve. The higher the measurement frequency, the closer we get to data that resembles the curve itself.
		
		In many cases functional data analysis restricts its scope to subsets of the functions $f:\mathbb{R} \rightarrow \mathbb{R}$.
		As these are inherently infinite-dimensional, it is necessary to introduce additional theory to appropriately deal with their unique properties. Sections \ref{Square_Integrable_Functions} and \ref{bases_L2} closely follow \cite{hsing_theoretical_2015} who provides a detailed introduction into the theory of functional data analysis.
		
		\begin{itemize}
			\item \cite{ramsay_functional_2005}
			\item \cite{kokoszka_introduction_2021}
		\end{itemize}
	
		\subsection{Hilbert Space of Square Integrable Functions}\label{Square_Integrable_Functions}
			\begin{definition}[Inner Product]
				A function $\langle \cdot , \cdot \rangle : \mathbb{V}^2 \rightarrow \mathbb{F}$ on a vector space $\mathbb{V}$ over a field $\mathbb{F}$ is called an inner product if the following four conditions hold for all $v, v_1, v_2 \in \mathbb{V}$ and $a_1, a_2 \in \mathbb{F}$.
				\begin{multicols}{2}
					\begin{enumerate}
						\item $\langle v,v \rangle \geq 0$
						\item $\langle v,v \rangle = 0$ if $v = 0$
						\item $\langle a_1 v_1 + a_2 v_2, v \rangle = a_1 \langle v_1, v \rangle + a_2 \langle v_2, v \rangle$
						\item $\langle v_1, v_2 \rangle = \overline{\langle v_2, v_1 \rangle}$
					\end{enumerate}
				\end{multicols}
			\end{definition}
			As this thesis is limited to the case $\mathbb{F} = \mathbb{R}$, property 4 can be restated as $\langle v_1, v_2 \rangle = \langle v_2, v_1 \rangle$, as the complex conjugate of a real number is the number itself. Similar to the case of Euclidean space, we say that two elements $v_1$ and $v_2$ of the inner product space are orthogonal if $\langle v_1, v_2 \rangle = 0$.
			
			\begin{definition}[Inner Product Space]
				A vector space with an associated inner product is called an inner product space.
			\end{definition}

			\begin{definition}[Hilbert Space]
				An inner product space that is complete with respect to the distance induced by the norm $\vert\vert v \vert\vert = \sqrt{\langle v, v\rangle}$ is called a Hilbert space.
			\end{definition}
		
			\begin{definition}[Closed Span]
				The closed span of a subset $A$ of some normed space, e.g. a normed vector space or a Hilbert space, is the closure of $\textit{span}\left(A\right)$ with respect to the distance induced by the norm of the space. In the following it is denoted by $\overline{{\textit{span}\left(A\right)}}$.
			\end{definition}
		
			{\color{red} This is verbatim!}
			\begin{definition}[Orthonormal Sequence in a Hilbert Space]
				Let $\{x_n\}$ be a countable collection of elements in a Hilbert space such that every finite subcollection of $\{x_n\}$ is linearly independent. Define $e_1 = \frac{x_1}{\vert\vert x_1 \vert\vert}$ and $e_i = \frac{v_i}{\vert\vert v_i \vert\vert}$ for 
				$$v_i = x_i - \sum_{j = 1}^{i - 1}\langle x_i, e_j\rangle e_j.$$
				Then, $\{e_n\}$ is an orthonormal sequence and $\overline{{\textit{span}\left(\{x_n\}\right)}} = \overline{{\textit{span}\left(\{e_n\}\right)}}$
			\end{definition}
		
			{\color{red} This is verbatim!}
			\begin{definition}[Orthonormal Basis of a Hilbert Space]
				An orthonormal sequence $\{e_n\}$ in a Hilbert space $\mathbb{H}$ is called an orthonormal basis of $\mathbb{H}$ if $\overline{{\textit{span}\left(\{e_n\}\right)}} = \mathbb{H}$.
			\end{definition}
		
			\begin{definition}[Separable Hilbert Space]
				A Hilbert space that possesses a countable complete orthonormal basis is called separable Hilbert space.
			\end{definition}
			Using the axiom of choice, it is possible to show that every Hilbert space possesses an orthonormal basis, which will be used in the derivation of an asymptotic distribution for a test statistic later in this thesis.
		
			\begin{definition}[Hilbert Space of Square Integrable Functions]
				
				The space of square integrable functions on a closed interval $A$ together with the norm $\langle f,g\rangle = \int_A f(t)g(t) \mathrm{d}t$ is a Hilbert space.
				A function $f: A \rightarrow \mathbb{R}$ is called square integrable if the following condition holds.
				\begin{equation}
					\int_{A} \left[f(t)\right]^2\mathrm{d}t < \infty
				\end{equation}
				The Hilbert space of all square integrable functions on $A$ is denoted by $\mathbb{L}_2(A)$.
			\end{definition}
			
			In most cases, $A$ is chosen as a closed interval of $\mathbb{R}$. So without loss of generality, we can reduce our treatment to the case of $A = [0,1]$.
			Deviating from the norm in functional data analysis, \cite{bugni_permutation_2021} define two functions to be distinct even if they differ only on a set of Lebesgue-measure zero.
		
		\subsection{Bases of $\mathbb{L}_2$}\label{bases_L2}
			
			One commonly used orthonormal basis of $\mathbb{L}^2([0,1])$ is the Fourier Basis. It consists of a series of functions $\left(\phi_{i}^{F}(x)\right)_{i \in \mathbb{N}}$ taken from the terms of the sine-cosine form of the Fourier series.
			\begin{equation}
				\phi_{i}^{F}(x) = 
				\begin{cases}
					1 & \text{if} \quad i = 1\\
					\sqrt{2} \cos(\pi i x) & \text{if} \quad i \quad \text{is even} \\
					\sqrt{2} \sin(\pi (i-1)x) & \text{otherwise}
				\end{cases}
			\end{equation}
			Figure \ref{fourier_basis} shows the first seven Fourier basis functions on $[0,1]$.
			\begin{figure}[H]
				\includegraphics[width = \textwidth]{../Graphics/fourier\_basis.PDF}
				\caption{The first seven Fourier basis functions}
				\label{fourier_basis}
			\end{figure}
			A proof that the Fourier basis is in fact an orthonormal basis of $\mathbb{L}^2([0,1])$ can be found in section 2.4 of \cite{hsing_theoretical_2015}. As the Fourier basis is a countable orthonormal basis of $\mathbb{L}^2([0,1])$, we can follow that $\mathbb{L}^2([0,1])$ is a separable Hilbert space, which will be useful in further parts of this thesis.
	
		\subsection{Random Functions}
			Random functions are a special case of general random variables. To understand their connection to the general concepts it is therefore useful to remind ourselves of the definition of a random variable. Paraphrasing from \cite{bauer_probability_2011} this can take the following form.
			\begin{definition}[Random Variable]
				Let $\left(\Omega, \mathcal{A}, \mathcal{P}\right)$ be a probability space and $\left(\Omega', \mathcal{A}'\right)$ be a measure space. Then every $\mathcal{A}$-$\mathcal{A}'$-measurable function $X:\Omega \rightarrow \Omega'$ is called a $\left(\Omega', \mathcal{A}'\right)$-random variable.
			\end{definition}
		
			\begin{definition}[Random Function]
				A random variable that realizes in a function space, e.g. $\mathbb{L}^2[0,1]$, is called a random function.
			\end{definition}
		
		\subsection{Probability Measures on $\mathbb{L}_2$}\label{prob_measures_l2}
			\begin{itemize}
				\item Kolmogorov Extension Theorem
				\item \cite{gihman_theory_2004}
			\end{itemize}
		
		\subsection{Functional Integration on $\mathbb{L}_2$}
			\begin{itemize}
				\item Perturbation theory
			\end{itemize}
		
			In one of the test statistics used in \cite{bugni_permutation_2021}, it is necessary to integrate over a function space. Therefore, it is necessary to explore the ideas of functional integration and integration on Hilbert spaces more general. Because of the special case relevant to the test statistic, integration on separable Hilbert spaces shall take a special focus.
			\begin{equation}
				\int_{\mathbb{L}_2(\mathcal{I})} G\left[f\right] \left[Df\right] = \int_{-\infty}^{\infty}\dots\int_{-\infty}^{\infty} G\left[f\right] \prod_{x} \mathrm{d}f(x)
			\end{equation}
		
			If a representation in terms of an orthogonal functional basis is possible:
			\begin{equation}
				\int_{\mathbb{L}_2(\mathcal{I})} G\left[f\right] \left[Df\right] = \int_{-\infty}^{\infty}\dots\int_{-\infty}^{\infty} G\left(f_1, f_2, \dots\right) \prod_{n} \mathrm{d}f_n
			\end{equation}
		
			An in-depth treatment of integration on Hilbert spaces is available in \cite{skorohod_integration_1974}.
		
	\section{Cram\'{e}r-von Mises Tests}\label{CvM_Tests}
		In applied econometrics, it is often interesting to ask whether the same stochastic process generated the observations in two distinct data sets. In an experimental setting, we could ask whether a treatment assigned at random to a subset of agents changed the distribution of an outcome variable. 
		One approach to answering this question is given by the two-sample Cram\'{e}r-von Mises test.
		
		\begin{itemize}
			\item \cite{darling_kolmogorov-smirnov_1957}
			\item \cite{anderson_asymptotic_1952}
			\item \cite{buning_nichtparametrische_2013}
		\end{itemize}
	
		\subsection{Empirical Distribution Functions}
			
			\cite{gibbons_nonparametric_2021}
			\begin{definition}[Order Statistic]\label{Order_Stat}
				
				Let $\{x_i \: \vert \: i = 1, \dots , n\}$ be a random sample from a population with continuous cumulative distribution function $F_X$. Then there almost surely exists a unique ordered arrangement within the sample. 
				
				$$X_{(1)} < X_{(2)} < \dots < X_{(n)}$$
				
				$X_{(r)} \quad r \in \{1, \dots, n\}$ is called the $r$th-order statistic.	
			\end{definition}
		
			\begin{definition}[Empirical Distribution Function]
				\begin{equation}
					F_{n}(x) = \begin{cases}
						0 & \quad \text{if} \quad  x < x_{(1)} \\
						\frac{r}{n} & \quad \text{if} \quad  x_{(r)} \leq x < x_{(r + 1)} \\
						1 & \quad \text{if} \quad  x \geq x_{(n)}
					\end{cases}
				\end{equation}
			\end{definition}
		
		\subsection{Assumptions}
		
		\subsection{Nullhypothesis}
			Let $\{x_1, \dots , x_n\}$ and $\{y_1, \dots , y_m\}$ be two data sets generated by random variables $X \sim_{\text{i.i.d.}} F(t)$ and $Y \sim_{\text{i.i.d.}} G(t)$.
			Then, we can formulate the Nullhypothesis that both samples were independently generated by random variables following the same distribution function.
			\begin{equation}
				\begin{split}
					H_0&: F(t) = G(t) \quad \forall t \in \mathbb{R}\\
					H_1&: \exists t \in \mathbb{R} \quad \text{s.t.} \quad F(t) \neq G(t)
				\end{split}
			\end{equation}
			
		\subsection{Two-Sample Cram\'{e}r-von Mises Statistic}
			
			\cite{buning_nichtparametrische_2013}
			\begin{equation}
				C_{m,n} = \left(\frac{nm}{n+m}\right) \int_{-\infty}^{\infty}\left(F_{m}(x) - G_{n}(x)\right)^{2} \mathrm{d} \left(\frac{m F_{m}(x) + n G_{n}(x)}{m+n}\right)
			\end{equation}
			\cite{anderson_distribution_1962} explores the small sample distribution of this test statistic and provides a comparison to the limiting distribution derived by \cite{rosenblatt_limit_1952} and \cite{fisz_result_1960}.
		
		\subsection{Asymptotic Distribution}
			As shown by the previously mentioned authors, under the Nullhypothesis that both samples were independently generated by random variables sharing the same distribution function, we can find the following limiting distribution of $C_{m,n}$.
			\begin{equation}
				\begin{split}
					C_{m,n} &\xrightarrow{\text{d}} \int_{0}^{1} \left(Z(u) + \left(1 + \lambda\right)^{-\frac{1}{2}} f(u) - \left[\frac{\lambda}{1+\lambda}\right]^{\frac{1}{2}}g(u)\right)^2 \mathrm{d}u \\
					\text{as} \quad &n \rightarrow \infty, \quad m \rightarrow \infty, \quad \frac{n}{m} \rightarrow \lambda \in \mathbb{R}
				\end{split}
			\end{equation}
			Here, $Z(u)$ is a Gaussian stochastic process with the following properties.
			\begin{itemize}
				\item $\mathbb{E}\left[Z(u)\right] = 0 \quad \forall u \in [0,1]$
				\item $Cov\left(Z(u), Z(v)\right) = \min(u,v) - uv \quad \forall u,v \in [0,1]$
			\end{itemize}					
			
	\section{Permutation Tests}\label{Permutation_Tests}
		In layman's terms, the idea of a permutation test is the following: if two samples show distinctly different properties, that will lead to differences in an appropriately chosen summary statistic. If we were to permute the elements of the groups randomly, we would expect these differences to disappear.
		Permutation tests formalize this intuition. The following section closely follows chapter 15 from \cite{lehmann_testing_2005}.
		
		\begin{itemize}
			\item \cite{van_der_vaart_weak_1996}
		\end{itemize}
	
		\subsection{Functional Principle of Permutation Tests}
		
			One on the defining features of each test is its Nullhypothesis. For the case of randomization tests, we can formulate it quite generally. Let $X$ be data taking values in a sample space $\mathcal{X}$. Then, the hypothesis is that the probability law $P$ generating $X$ belongs to a family of distributions $\Omega_0$. \cite{lehmann_testing_2005} define an assumption called the randomization hypothesis that allows for the construction of randomization tests. As permutation tests are a special case of randomization tests, we can specialize this definition to the case under consideration.
			\begin{assumption}[Randomization Hypothesis]\label{rand_hypo}
				 Let $G$ be a finite group of transformations $g: \mathcal{X} \rightarrow \mathcal{X}$. Under the Nullhypothesis of the randomization test, the distribution of $X$ is invariant under the transformations $g \in G$. In other words, $gX$ and $X$ have the same distribution whenever $X$ has distribution $P \in \Omega_0$.
			\end{assumption}
			Under this assumption one can construct a permutation test based on any test statistic $T:\mathcal{X} \rightarrow \mathbb{R}$ that is suitable to test the Nullhypothesis under consideration. Suppose that $G$ has $M$ elements, then given $X = x$, let 
			$$T_{(1)}(x) \leq T_{(2)}(x) \leq \dots \leq T_{(M)}(x) $$
			be the ordered values of the test statistic $T(gx)$ as described in Definition \ref{Order_Stat} as $g$ varies over $G$. For a fixed nominal level $\alpha \in (0,1)$, define $k = M - \lfloor M\alpha \rfloor$. Additionally define the following two objects.
			\begin{multicols}{2}
				\noindent
				\begin{equation*}
					M^{+} = \sum_{m = 1}^{M}\mathbbm{1}\left[T_{(m)}(x) > T_{(k)}(x)\right]
				\end{equation*}
				\begin{equation}
					M^{0} = \sum_{m = 1}^{M}\mathbbm{1}\left[T_{(m)}(x) = T_{(k)}(x)\right]
				\end{equation}
			\end{multicols}
			
			\begin{definition}[Randomization Test Function]\label{RandTestFunc}
				We define the Randomization Test Function as the following function $\phi: \mathcal{X} \rightarrow \mathbb{R}$.
					\begin{equation*}
						\phi(x) = \begin{cases}
							1 &\text{if} \ T > T_{(k)}(x) \\
							a &\text{if} \ T = T_{(k)}(x) \\
							0 &\text{if} \ T < T_{(k)}(x) \\
						\end{cases} \quad \text{where} \quad
						a = \frac{M\alpha - M^{+}(x)}{M^{0}(x)}
					\end{equation*}
				
			\end{definition}
			Under Assumption \ref{rand_hypo}, it is possible to show that, given a test statistic $T = T(X)$, the resulting test $\phi$ has size $\alpha$.
			\begin{equation}
				\mathbb{E}_{P}\left[\phi(X)\right] = \alpha \quad \forall P \in \Omega_0
			\end{equation}
			
		
			\cite{lehmann_testing_2005} explore the example of testing for the equality of the generating probability laws of two independent samples. This is precisely the relevant application for the permutation variant of the two-sample Cram\'{e}r-von Mises test that \cite{bugni_permutation_2021} extend to the setting of functional data. \\
			
			\begin{definition}[Permutation]
				Let $S$ be a set, then a permutation of $S$ is a bijective function $\pi: S \rightarrow S$.
			\end{definition}
			If $S$ is a finite set with $N$ elements, there are $N!$ different permutations. If we apply this idea to the setting of two samples with $n$ and $m$ observations respectively, there are $(n+m)!$ permutations in the combined set of observations. One way of describing the corresponding group of transformations $G$ is shown in Equation \ref{Permutations}.
			\begin{equation}\label{Permutations}
				\begin{split}
					\Pi_N &= \left\{\pi: \left\{1, \dots, N \right\} \rightarrow \left\{1, \dots, N \right\} \ \vert \ \pi \ \text{is bijective} \right\} \\
					G &= \left\{g:\mathbb{R}^N \rightarrow \mathbb{R}^N \ \vert \ \exists \pi \in \Pi_N \ \forall x \in \mathbb{R}^N \ g(x) = \left(x_{\pi(1)}, \dots, x_{\pi(N)}\right) \right\}
				\end{split}
			\end{equation}
			
		 
			Number of Combinations: $\binom{m+n}{m}$\\
			
			For my implementation, I chose the latter variant.
	
		\subsection{Size and Power}
		
	\section{Test by Bugni and Horowitz (2021)}\label{Bugni_Horowitz_2021}
	
		In \cite{bugni_permutation_2021}, the authors devise a permutation test for the equality of the distribution of two samples of functional data. To define the exact hypothesis, it is therefore necessary to define a distribution function for a random variable realizing in $\mathbb{L}_2(\mathcal{I})$.
		\begin{definition}[Distribution Function of a Random Function]
			Let $X:\Omega \rightarrow \mathbb{L}_2(\mathcal{I})$ be a random function realizing in the square-integrable functions. Then its distribution function is defined as the following object.
			\begin{equation*}
				F_X(z) = \mathbb{P}\left[X(t) \leq z(t) \quad \forall t \in \mathcal{I}\right] \quad z \in \mathbb{L}_2(\mathcal{I})
			\end{equation*}
		\end{definition}
		Deviating from the norm in functional data analysis, the authors assume that two functions $z_1, z_2 \in \mathbb{L}_2(\mathcal{I})$ are distinct even if they only differ on a set of Lebesgue-measure zero.\footnote{Does this create a problem about the Fourier basis being a complete orthonormal basis of the space?}
		
		\subsection{Assumptions}
		
			\begin{assumption} Contains two assumptions
				\begin{enumerate}
					\item $X(t)$ and $Y(t)$ are separable, $\mu$-measurable stochastic processes.
					\item $\{X_i(t) \: \vert \: i = 1, \dots, n\}$ is an independent random sample of the process $X(t)$. \\
					$\{Y_i(t) \: \vert \: i = 1, \dots, m\}$ is an independent random sample of $Y(t)$ and is independent of $\{X_i(t) \: \vert \: i = 1, \dots, n\}$.
				\end{enumerate}
			\end{assumption}
		
			\begin{definition}[Separable Stochastic Process]
				{\color{red} From Wikipedia:} A real-valued continuous time stochastic process $X$ with a probability space $\left(\Omega, \mathcal{F}, \mathcal{P}\right)$ is separable if its index set $T$ has a dense countable subset $U \subset T$ and there is a set $\Omega_0 \subset \Omega$ of probability zero, so $\mathcal{P}\left(\Omega_0\right) = 0$, such that for every open set $G \subset T$ and every closed set $F \subset \mathbb{R}$ the two events $\left\{X_t \in F \quad \forall t \in G \cap U\right\}$ and $\left\{X_t \in F \quad \forall t \in G\right\}$ differ from each other at most on a subset $\Omega_0$.
			\end{definition}
		
			In less theoretical terms this means that the process is determined by its values on a countable subset of points of its index set.
		
			\begin{assumption}
				$\mathbb{E}X(t)$ and $\mathbb{E}Y(t)$ exist and are finite for all $t \in [0, T]$.
			\end{assumption}
		
			\begin{assumption}\label{continuous_observation}
				$X_i(t)$ and $Y_i(t)$ are observed for all $t \in \mathcal{I}$.
			\end{assumption}
			Assumption \ref{continuous_observation} can be relaxed and a similar test can be constructed for the case of discretely observed processes. This variation of the test will not be addressed in this thesis. However, \cite{bugni_permutation_2021} provide a description of how to extend their idea to this common scenario.
	
		\subsection{Nullhypothesis}
			\begin{equation}
				\begin{split}
					H_0: \quad &F_X(z) = F_Y(z) \quad \forall z \in \mathbb{L}_2(\mathcal{I}) \\
					H_1: \quad &\mathbb{P}_{\mu}\left[F_X(Z) \neq F_Y(Z)\right] > 0
				\end{split}
			\end{equation}
			Here, $\mu$ is a probability measure on $\mathbb{L}_2(\mathcal{I})$ and $Z$ is a random function with probability distribution $\mu$. {\color{red} Doesn't this leave out the case where the Probability functions only differ on a set of $\mu$-measure zero?}
		
		\subsection{Cram\'{e}r-von Mises type Test}
		
			Empirical Distribution Functions
			\begin{multicols}{2}
				\noindent
				\begin{equation*}
					\hat{F}_X(z) = \frac{1}{n} \sum_{i = 1}^{n}\mathbbm{1}\left[X_i(t) \leq z(t) \ \forall t \in \mathcal{I}\right]
				\end{equation*}
				\begin{equation}
					\hat{F}_Y(z) = \frac{1}{m} \sum_{i = 1}^{m}\mathbbm{1}\left[Y_i(t) \leq z(t) \ \forall t \in \mathcal{I}\right]
				\end{equation}
			\end{multicols}
			
			
			Test statistic
			\begin{equation}
				\tau = \int_{\mathbb{L}_2(\mathcal{I})}\left[F_X(z) - F_Y(z)\right]^2 \mathrm{d} \mu(z)
			\end{equation}
			
			Sample analog:
			\begin{equation}
				\tau_{n,m} = (n+m) \int_{\mathbb{L}_2(\mathcal{I})}\left[\hat{F}_X(z) - \hat{F}_Y(z)\right]^2 \mathrm{d} \mu(z)
			\end{equation}
		
			Critical values for Permutation Test Statistic
			\begin{equation}
				t^{*}_{n,m}(1-\alpha) = \inf \left\{t \in \mathbb{R} \quad \vert \quad \frac{1}{Q} \sum_{i = 1}^{Q} \mathbbm{1}\left[\tau_{n,m,q} \leq t\right] \geq 1 - \alpha \right\}
			\end{equation}
		
		\subsection{Asymptotics for the Cram\'{e}r-von Mises type Test}
			Similar to the case presented in \cite{bugni_goodness--fit_2009}, we can derive an asymptotic distribution for the Cram\'{e}r-von Mises type test. Even though this is not necessary to perform the described permutation test, it is an interesting benchmark to compare the test procedure.
		
		\subsection{Construction of the Measure $\mu$}
			As introduced in Section \ref{prob_measures_l2}, it is possible to construct a probability measure on the space of square integrable functions. For the calculation of the Cram\'{e}r-von Mises type test, we need to construct one such probability measure that is suitable to detect the kind of alternative we expect to find.
			\cite{bugni_permutation_2021} approach this problem by first constructing a random function that induces a probability measure. This random function is chosen using a function $w(t):\mathcal{I} \rightarrow \mathbb{R}$ that is supposed to be chosen large in the parts of $\mathcal{I}$ where possible differences between the empirical distribution functions are expected to be large.
			
			\begin{equation}\label{non_truncated}
				Z(t) = \sum_{k = 1}^{\infty} b_k \psi_k(t)
				\quad \text{s.t.} \quad
				\sum_{k = 1}^{\infty} b_k^2 < \infty \quad \text{a.s.}
			\end{equation}

			\begin{equation}\label{truncation}
				Z_K(t) = \sum_{k = 1}^{K} b_k \psi_k(t)
			\end{equation}
		
			\begin{equation}
				\mathbb{E}\left[Z_K(t)\right] = \sum_{k = 1}^{K} \mathbb{E}\left[b_k\right] \psi_k(t)
				\quad \text{where} \quad
				\mathbb{E}\left[b_k\right] = \int_{\mathcal{I}}w(t)\psi_k(t) \mathrm{d}t
			\end{equation}
		
			\begin{equation}
				b_k = \mathbb{E}\left[b_k\right] + \rho_k U_k
				\quad \text{s.t.} \quad
				\sum_{k = 1}^{\infty} \rho_k^2 < \infty 
			\end{equation}
			
			In \cite{bugni_goodness--fit_2009}, the authors show that the approximation of the probability measure, which is induced by the approximation of the random variable in Equation \ref{truncation}, converges appropriately to the probability measure induced by the random variable in Equation \ref{non_truncated}. Here, appropriately means, that integrals with respect to the approximation, henceforth called $\mu_K$, converge to their corresponding integral evaluated with respect to $\mu$.
		\subsection{Mean focused Test}
			
			Test statistic
			\begin{equation}
				\nu = \int_{\mathcal{I}} \big(\mathbb{E}\left[X(t)\right] - \mathbb{E}\left[Y(t)\right]\big)^2 \mathrm{d}t
			\end{equation}
		
			Mean Estimators
			\begin{multicols}{2}
				\noindent
				\begin{equation*}
					\hat{\mathbb{E}}\left[X(t)\right] = \frac{1}{n}\sum_{i = 1}^{n} X_i(t)
				\end{equation*}
				\begin{equation}
					\hat{\mathbb{E}}\left[Y(t)\right] = \frac{1}{m}\sum_{i = 1}^{m} Y_i(t)
				\end{equation}
			\end{multicols}
		
			Sample Analog
			\begin{equation}
				\nu_{n,m} = (n+m) \int_{\mathcal{I}} \left[\hat{\mathbb{E}}X(t) - \hat{\mathbb{E}}Y(t)\right]^2 \mathrm{d}t
			\end{equation}
		
			Critical values for Permutation Test Statistic
			\begin{equation}
				t^{*}_{n,m}(1-\alpha) = \inf \left\{t \in \mathbb{R} \quad \vert \quad \frac{1}{Q} \sum_{i = 1}^{Q} \mathbbm{1}\left[\nu_{n,m,q} \leq t\right] \geq 1 - \alpha \right\}
			\end{equation}
		
		\subsection{Combined Permutation Test}
			Define for the two underlying tests the following permutation test functions as described in Definition \ref{RandTestFunc} for the general case of a randomization test.
			\begin{multicols}{2}
				\noindent
				\begin{equation*}
					\phi_{n,m} = \begin{cases}
						1 &\text{if} \ \tau_{n,m} > t^{*}_{n,m}(1-\alpha_{\tau}) \\
						a_{\tau} &\text{if} \ \tau_{n,m} = t^{*}_{n,m}(1-\alpha_{\tau}) \\
						0 &\text{if} \ \tau_{n,m} < t^{*}_{n,m}(1-\alpha_{\tau}) \\
					\end{cases}
				\end{equation*}
				\begin{equation}
					\tilde{\phi}_{n,m} = \begin{cases}
						1 &\text{if} \ \nu_{n,m} > t^{*}_{n,m}(1-\alpha_{\nu}) \\
						a_{\nu} &\text{if} \ \nu_{n,m} = t^{*}_{n,m}(1-\alpha_{\nu}) \\
						0 &\text{if} \ \nu_{n,m} < t^{*}_{n,m}(1-\alpha_{\nu}) \\
					\end{cases}
				\end{equation}
			\end{multicols}
			$a_\tau$ and $a_\nu$ are given by the following equations to ensure that the expected values of $\phi$ and $\tilde{\phi}$ have the desired values.
			\begin{multicols}{2}
				\begin{itemize}
					\item $a_{\tau} = \frac{Q\alpha_{\tau} - Q_{\tau}^{+}}{Q_{\tau}^{0}}$ 
					\item $Q_{\tau}^{+} = \sum_{q = 1}^{Q}\mathbbm{1}\left[\tau_{n,m,q} > t^{*}_{n,m}(1-\alpha_{\tau})\right]$
					\item $Q_{\tau}^{0} = \sum_{q = 1}^{Q}\mathbbm{1}\left[\tau_{n,m,q} = t^{*}_{n,m}(1-\alpha_{\tau})\right]$
					\item $a_{\nu} = \frac{Q\alpha_{\nu} - Q_{\nu}^{+}}{Q_{\nu}^{0}}$ 
					\item $Q_{\nu}^{+} = \sum_{q = 1}^{Q}\mathbbm{1}\left[\nu_{n,m,q} > t^{*}_{n,m}(1-\alpha_{\nu})\right]$
					\item $Q_{\nu}^{0} = \sum_{q = 1}^{Q}\mathbbm{1}\left[\nu_{n,m,q} = t^{*}_{n,m}(1-\alpha_{\nu})\right]$
				\end{itemize} 
			\end{multicols}
			
			Bonferroni inequality under $H_0$ leads to
			\begin{equation}
				\max(\alpha_{\tau}, \alpha_{\nu}) \leq \mathbb{P}\left[(\phi_{n,m} > 0) \cup (\tilde{\phi}_{n,m} > 0)\right] \leq \alpha_{\tau} + \alpha_{\nu}
			\end{equation}
		
		\subsection{Finite Sample Properties under the Nullhypothesis}
		For any distribution $P$ that satisfies the Nullhypothesis and any $\alpha_{\tau}, \alpha_{\mu} \in (0,1)$, we have 
		\begin{multicols}{2}
			\noindent
			\begin{equation*}
				\mathbb{E}_P\left(\phi_{n,m}\right) = \alpha_{\tau}
			\end{equation*}			
			\begin{equation}
				\mathbb{E}_P\left(\tilde{\phi}_{n,m}\right) = \alpha_{\nu}
			\end{equation}
		\end{multicols}
		
		\subsection{Asymptotic Properties under the Alternative}
		
	\section{Closed Testing Procedures}
		
	\section{Variant for Alternatives in Specific Frequency Ranges}
		
		
	\section{Simulation Study}\label{Simulation_Study}
		\subsection{Implementation as an R package}
			All analyses in this thesis have been conducted with R\footcite{R}. I implemented the two-sample variant of the test presented taken from \cite{bugni_permutation_2021} in an R package called \textit{PermFDATest}. The R package and all code that has been used to produce the following results are publicly available as part of a GitHub repository\footnote{\href{https://github.com/JakobJuergens/Masters_Thesis}{https://github.com/JakobJuergens/Masters\_Thesis}} that complements this thesis.
			
			\begin{itemize}
				\item \cite{fda}
				\item \cite{tidyverse}
				\item \cite{refund}
			\end{itemize}
			
		\subsection{Use of High-Performance Computing}
			The simulations presented as part of this thesis have been conducted on \textit{bonna}\footnote{\href{https://www.dice.uni-bonn.de/de/hpc/hpc-a-bonn/infrastruktur}{https://www.dice.uni-bonn.de/de/hpc/hpc-a-bonn/infrastruktur}}. \textit{bonna} is the high performance computing cluster provided by the University of Bonn. The implementation is heavily parallelized and makes use of a SLURM scheduling system. However, slight modifications of the provided code suffice to run it on personal computers.

		\subsection{Simulation Setup}
		
		\subsection{Results}
		
	\section{Application}\label{Application}
	
	\section{Outlook}\label{Outlook}
	
	\newpage
	\section{Bibliography}
	\printbibliography[heading=none]
	
	\newpage
	\cleardoublepage
	\pagenumbering{roman}
	\setcounter{page}{1}
	\section{Appendix}
	
		\subsection{Multiple Testing}\label{Multiple_Testing}
			When testing statistical hypotheses, it is often helpful or even necessary to test multiple hypotheses independently of each other. One setting where this could be useful is when we want to combine the desirable properties of two tests, as is done by \cite{bugni_permutation_2021}. If the tests do not perfectly depend on each other, this creates a problem relating to the size of the combined test.
			
			\begin{definition}[Family-wise Error Rate]
					content...
				\end{definition}
		
				The most straightforward correction for this multiple testing problem is the so-called Bonferroni Correction. Introduced by \cite{dunn_multiple_1961}, it is based on Boole's Inequality, which is sometimes referred to as the Bonferroni Inequality.
				\begin{equation}
						\mathbb{P}\left[\bigcup_{i = 1}^{\infty} A_i\right] \leq \sum_{i = 1}^{\infty} \mathbb{P}\left[A_i\right]
					\end{equation}
				for a countable set of events $A_1, A_2, \dots$.
	
	\newpage
	\thispagestyle{empty}
	\section*{Versicherung an Eides statt}	
	
		\vspace{3cm}
		
		Ich versichere hiermit, dass ich die vorstehende Masterarbeit
		selbstständig verfasst und keine anderen als die angegebenen Quellen
		und Hilfsmittel benutzt habe, dass die vorgelegte Arbeit noch an keiner
		anderen Hochschule zur Prüfung vorgelegt wurde und dass sie weder
		ganz noch in Teilen bereits veröffentlicht wurde. Wörtliche Zitate und
		Stellen, die anderen Werken dem Sinn nach entnommen sind, habe ich
		in jedem einzelnen Fall kenntlich gemacht.
		
		\vspace{2cm}
		Bonn, XX.XX.2021 \hrulefill \\
		\hspace*{0mm}Jakob R. Juergens
		
		\vspace{\fill}
\end{document}